%%%%%%%%%%%%%%%%%%%%%%%%%%%%%%%%%%%%%%%
% Deedy CV/Resume
% XeLaTeX Template
% Version 1.0 (5/5/2014)
%
% This template has been downloaded from:
% http://www.LaTeXTemplates.com
%
% Original author:
% Debarghya Das (http://www.debarghyadas.com)
% With extensive modifications by:
% Vel (vel@latextemplates.com)
%
% License:
% CC BY-NC-SA 3.0 (http://creativecommons.org/licenses/by-nc-sa/3.0/)
%
% Important notes:
% This template needs to be compiled with XeLaTeX.
%
%%%%%%%%%%%%%%%%%%%%%%%%%%%%%%%%%%%%%%

\documentclass[letterpaper]{deedy-resume} % Use US Letter paper, change to a4paper for A4 

\begin{document}

%----------------------------------------------------------------------------------------
%	TITLE SECTION
%----------------------------------------------------------------------------------------


\namesection{Leo}{Neat}{ % Your name
\href{mailto:lneat@ucsc.edu}{lneat@ucsc.edu} | 818.331.6178 % Your contact information
}

%----------------------------------------------------------------------------------------
%	LEFT COLUMN
%----------------------------------------------------------------------------------------

\begin{minipage}[t]{0.33\textwidth} % The left column takes up 33% of the text width of the page

%------------------------------------------------
% Education
%------------------------------------------------

\section{Education} 

\subsection{UC Santa Cruz }



\descript{BS in Computer Science}
\location{Expected May 2019 | Santa Cruz, CA}
Conc. in Software Engineering \\
Jack Baskin School of Engineering \\
Dean's List (All Semesters) \\
\location{ Cum. GPA: 3.92 / 4.0}

\descript{(Awaiting Acceptance)}
\descript{M.S. in Computer Engineering}
\location{Expected May 2020 | Santa Cruz, CA}
4 $+$ 1 B.S/M.S. program\\
Currently taking graduate level classes to apply to my M.S. degree upon acceptance 

\sectionspace % Some whitespace after the section


\sectionspace % Some whitespace after the section

%------------------------------------------------
% Links
%------------------------------------------------
\iffalse
\section{Links} 

Github:// \href{https://github.com/Leo-Neat}{\bf Leo-Neat} \\
LinkedIn:// \href{https://www.linkedin.com/in/leo-neat-82b34b130/}{\bf leo-neat} \\

\sectionspace % Some whitespace after the section
\fi
%------------------------------------------------
% Coursework
%------------------------------------------------

\section{Coursework}

\subsection{Graduate}

Computer Vision and Image Processing \\
Advanced Algorithm Analysis \\\
%------------------------------------------------
\subsection{Undergraduate}

Machine Learning \\
Advanced Programming \\
Comparative Programming Languages \\
Abstract Data Types \\
Web Development \\
Probability and Statistics \\
Computer Systems \\
Assembly Language \\
Discrete Math \\
Data Structures \\
Linear Algebra \\
Vector Calculus \\


\sectionspace % Some whitespace after the section

%------------------------------------------------
% Skills
%------------------------------------------------

\section{Skills}

\subsection{Programming}

\location{Over 5000 lines:}
Java \textbullet{} Python \textbullet{} Android \\ 
\location{Over 1000 lines:}
C \textbullet{} C++ \\
\location{Familiar:}
 Shell  \textbullet{} Arduino

\subsection{Tools}
\location{Experienced:}
	Android Studio \textbullet{} Linux O.S. \textbullet{} \LaTeX \textbullet Git \\
\location{Intermediate}
	OpenCV \textbullet{} SQLite \textbullet{} Keras \\
\location{Familiar}
	AWS \textbullet{} Google Firebase \textbullet{} TensorFlow's Object Detection API \textbullet{} Cmake

\sectionspace % Some whitespace after the section

%----------------------------------------------------------------------------------------

\end{minipage} % The end of the left column
\hfill
%
%----------------------------------------------------------------------------------------
%	RIGHT COLUMN
%----------------------------------------------------------------------------------------
%
\begin{minipage}[t]{0.66\textwidth} % The right column takes up 66% of the text width of the page

%------------------------------------------------
% Experience
%------------------------------------------------

\section{Android Development Experience}

%------------------------------------------------------
% Computer Vision Lab Devlopment
%------------------------------------------------------
\runsubsection{UCSC Computer Vision Lab }\\
\descript{Onboard Inference Speed Test Application}
\location{September 2018 – present| Santa Cruz, CA}
\vspace{\topsep}
\begin{tightitemize}
	\item Developed an android application that can be used to compare the inference  speed of different mobile Convolutional Neural Networks(CNNs)
	\item Application synchronously detects objects in the camera's field of and display's bounding boxes over the Camera2 API preview
	\item Allows lab students to upload their own TensorFlow Lite CNN to test and compare inference speeds.
\end{tightitemize}

\descript{CNN Model Test Application}
\location{December 2016 – June 2018| Santa Cruz, CA}
\begin{tightitemize}
	\item Worked with a P.H.D. student to develop an assistive technology application to aid blind individual's spacial awareness 
	\item Developed a system that streamed images from an Android device's camera to a server where it was processed by a text detection CNN
	\item Created an Android UX that was specifically used to relay the information from the server to blind individuals
	\item A paper about the application was accepted for publication: \\
	L. Neat, R. Peng, S. Qin, R. Manduchi "Scene Text Access: A Comparison of  Mobile OCR Modalities for Blind Users" 23rd International Conference on Intelligent User Interfaces. ACM, 2019
	
\end{tightitemize}

%-----------------------------------------------------------------
% Aquifi App Devlopment
%-----------------------------------------------------------------
\runsubsection{Aquifi}
\descript{| Software Engineering Intern}
\location{June 2018 – Sep 2018 | Palo Alto, CA}
 % Hacky fix for awkward extra vertical space
\begin{tightitemize}
	\item Devloped an Android application that was used to off-load data from Aquifi's handheld camera devices
	\item System would send user selected image data and meta-data to different Aquifi servers for regression testing, training, etc...
	\item Used AWS SNS to notify respective company teams when new data was uploaded
	\item Application is still being used by Aquifi today
\end{tightitemize}
\sectionspace % Some whitespace after the section

%-----------------------------------------------------------------
% JPL App Devlopment
%-----------------------------------------------------------------
\runsubsection{Jet Propulson Laboratory }
\descript{| Software Engineering Intern}
\location{June 2017 – Sep 2017| Pasadena, CA}
	\begin{tightitemize}
			\item 	Designed and built a software and hardware system that utilizes an Android phone to emulate a star for EMCCD camera testing
			\item Saved NASA thousands of dollars in development cost and months of testing
			\item The Android application allowed JPL scientists to image complex light structures that were emitted by the phone screen
			\item 	The camera characterization system is currently being used by different teams at NASA for spaceflight detector evaluation 
			\item The system was published in a well known Astronomy journal:\\
				Michael Bottom, Leo S. Neat, Leon K. Harding, Patrick Morrissey, Seth R. Meeker, Richard T. Demers  "Smartphone scene generator for efficient characterization of visible imaging detectors", Proc. SPIE 10709, High Energy, Optical, and Infrared Detectors for Astronomy VIII, 107092R (6 July 2018); doi: 10.1117/12.2312335;
	\end{tightitemize}

\sectionspace

%-----------------------------------------------------------------
% Personal App Devlopment
%-----------------------------------------------------------------
\runsubsection{Personal }
\descript{|On-board Assistive Text Detector}
\location{September 2018 – present| Santa Cruz, CA}
\begin{tightitemize}
	\item Developing an Android app for blind individuals inspired by my work at UCSC
	\item Application uses Google Firebase Vision detection as its text detector
	\item Currently in alpha testing but will be pushed to the Google Play Store soon 
\end{tightitemize}
\end{minipage} % The end of the right column
\end{document}
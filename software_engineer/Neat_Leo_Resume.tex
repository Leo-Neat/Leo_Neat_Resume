%%%%%%%%%%%%%%%%%%%%%%%%%%%%%%%%%%%%%%%
% Deedy CV/Resume
% XeLaTeX Template
% Version 1.0 (5/5/2014)
%
% This template has been downloaded from:
% http://www.LaTeXTemplates.com
%
% Original author:
% Debarghya Das (http://www.debarghyadas.com)
% With extensive modifications by:
% Vel (vel@latextemplates.com)
%
% License:
% CC BY-NC-SA 3.0 (http://creativecommons.org/licenses/by-nc-sa/3.0/)
%
% Important notes:
% This template needs to be compiled with XeLaTeX.
%
%%%%%%%%%%%%%%%%%%%%%%%%%%%%%%%%%%%%%%

\documentclass[letterpaper]{deedy-resume} % Use US Letter paper, change to a4paper for A4 

\begin{document}

%----------------------------------------------------------------------------------------
%	TITLE SECTION
%----------------------------------------------------------------------------------------


\namesection{Leo}{Neat}{ % Your name
\href{mailto:lneat@ucsc.edu}{lneat@ucsc.edu} | 818.331.6178 | Santa Cruz, CA % Your contact information
}

%----------------------------------------------------------------------------------------
%	LEFT COLUMN
%----------------------------------------------------------------------------------------

\begin{minipage}[t]{0.30\textwidth} % The left column takes up 33% of the text width of the page

%------------------------------------------------
% Education
%------------------------------------------------

\section{Education} 

\subsection{UC Santa Cruz }

\descript{BS in Computer Science}
\location{Expected May 2019}
Dean's List (All Semesters) \\
\location{ Cum. GPA: 3.92 / 4.0}
\sectionspace

\descript{M.S. in Computer Engineering}
\descript{(Awaiting Acceptance)}
\location{Expected May 2020}
4 $+$ 1 B.S/M.S. program\\
Currently taking graduate level classes to apply to my M.S. degree upon acceptance 
\sectionspace % Some whitespace after the section

%------------------------------------------------
% Links
%------------------------------------------------

\
%------------------------------------------------
% Coursework
%------------------------------------------------

\section{Coursework}

\subsection{Graduate}

Computer Vision and Image Processing \\
Advanced Algorithm Analysis \\\
%------------------------------------------------
\subsection{Undergraduate}


Machine Learning \\
Advanced Programming \\
Comparative Programming Languages \\
Probability and Statistics \\
Computer Architecture \\ 
Computer Systems \\
Assembly Language \\
Web Development \\
Discrete Math \\
Data Structures \\
Communication in C.S. \\
Linear Algebra \\
Vector Calculus \\


\sectionspace % Some whitespace after the section

%------------------------------------------------
% Programming Languages
%------------------------------------------------

\section{Languages}


\subsection{Expert}
Java \textbullet{} Python \textbullet{} Android \\ 
\subsection{Proficient}
C \textbullet{} C++ \\
\subsection{Familiar}
 Shell  \textbullet{} SQLite \textbullet{} Scheme \textbullet{} Matlab \textbullet{} XML \textbullet{} Java Script 

\sectionspace % Some whitespace after the section

%----------------------------------------------------------------------------------------

\section{Links} 

Github:// \href{https://github.com/Leo-Neat}{\bf Leo-Neat} \\
LinkedIn:// \href{https://www.linkedin.com/in/leo-neat-82b34b130/}{\bf leo-neat} \\

\sectionspace % Some whitespace after the section

\end{minipage} % The end of the left column
\hfill
%
%----------------------------------------------------------------------------------------
%	RIGHT COLUMN
%----------------------------------------------------------------------------------------
%
\begin{minipage}[t]{0.66\textwidth} % The right column takes up 66% of the text width of the page

%------------------------------------------------
% Experience
%------------------------------------------------

\section{Experience}

\runsubsection{UC Santa Cruz Computer Vision Lab}
\descript{| Undergrad Research}

\location{December 2016 – Present | Santa Cruz, CA}
\vspace{\topsep} % Hacky fix for awkward extra vertical space
\begin{tightitemize}
	\item Developed a Convolutional Neural Network testing pipeline by implementing an android client to stream  live camera data to a Linix server for processing. This resulted in quick and effective CNN evaluations for the lab. 
	\item Developed an Android application to help the visually impaired recognize text in their surroundings which led to the following publication: "Scene Text Access: A Comparison of Mobile OCR Modalities for Blind Users" 23rd International Conference on Intelligent User Interfaces. ACM, 2019.
	\item Developed a system to test the inference speed of TensorFlow Lite models on mobile devices in order to determine costs and benefits of on-board vs. server side inference for mobile CNNs.
\end{tightitemize}
\sectionspace % Some whitespace after the section

\runsubsection{Aquifi}
\descript{| Software Engineering Intern}

\location{June 2018 – Sep 2018 | Palo Alto, CA}
\begin{tightitemize}
	\item Developed an Android application that allowed for users to quickly offload data from Aquifi devices to Aquifi servers for quality assurance, regression testing, and training.
	\item Created an algorithm to detect corrupted frames in Aquifi camera stream, which resulted in the assurance that Aquifi devices were initialized correctly.
\end{tightitemize}

\sectionspace % Some whitespace after the section

%------------------------------------------------

\runsubsection{Jet Propulson Laboratory }
\descript{| Software Engineering Intern}
\location{June 2014 – Sep 2017 | Palo Alto, CA}
\begin{tightitemize}
	\item 	Designed, developed and built an Optomechanical System that utilizes an Android phone to emulate a star for space camera testing. This resulted in saving the WFIRST detector team months of camera testing, thousand of dollars, and the following publication: "Smartphone scene generator for efficient characterization of visible imaging detectors", Proc. SPIE 10709, High Energy, Optical, and Infrared Detectors for Astronomy VIII.
	
	\item Parallelized the initialization process of a telescope testbed which resulted in the reduction of the initialization time by a factor of seven.   
	
\end{tightitemize}
\sectionspace % Some whitespace after the section


%------------------------------------------------
% Projects
%------------------------------------------------
\section{Projects}

%________________ ASSISTIVE TEXT DETECTOR _______________________
\runsubsection{Assistive Text Detector}
\descript{| Personal Project}
\location{Sept 2018 – Present}
\begin{tightitemize}
	\item Created an Android application that assists the visually impaired in the navigation of the world by preforming live on-board text detection and optical character recognition. This application is in internal testing phase and will be available on the google play store in the next couple of weeks.  
\end{tightitemize}

\sectionspace % Some whitespace after the secti
%_________________ CROWD SIZE DETECTOR___________________________
\runsubsection{Crowd Size Detector}
\descript{| Personal Project}
\location{Dec 2017 – Present}
\begin{tightitemize}
	\item Currently developing a  platform using TensorFlow's Object Detection API to monitor the number of people in a variety of public locations. 
	\item Plan to notify users via AWS's SNS when crowd levels are low. 
\end{tightitemize}

\sectionspace

%________________CRUZ HACKS_______________________________________
\runsubsection{Cruz Hacks}
\descript{| Hackathon}
\location{Jan 2017 }
\begin{tightitemize}
	\item Developed a Monte Carlo simulation to help predict the demand of every computer science class at UC Santa Cruz. 
\end{tightitemize}



%----------------------------------------------------------------------------------------

\end{minipage} % The end of the right column
\end{document}
%%%%%%%%%%%%%%%%%%%%%%%%%%%%%%%%%%%%%%%
% Deedy CV/Resume
% XeLaTeX Template
% Version 1.0 (5/5/2014)
%
% This template has been downloaded from:
% http://www.LaTeXTemplates.com
%
% Original author:
% Debarghya Das (http://www.debarghyadas.com)
% With extensive modifications by:
% Vel (vel@latextemplates.com)
%
% License:
% CC BY-NC-SA 3.0 (http://creativecommons.org/licenses/by-nc-sa/3.0/)
%
% Important notes:
% This template needs to be compiled with XeLaTeX.
%
%%%%%%%%%%%%%%%%%%%%%%%%%%%%%%%%%%%%%%

\documentclass[letterpaper]{deedy-resume} % Use US Letter paper, change to a4paper for A4 

\begin{document}

%----------------------------------------------------------------------------------------
%	TITLE SECTION
%----------------------------------------------------------------------------------------


\namesection{Leo}{Neat}{ % Your name
\href{mailto:lneat@ucsc.edu}{lneat@ucsc.edu} | 818.331.6178 % Your contact information
}

%----------------------------------------------------------------------------------------
%	LEFT COLUMN
%----------------------------------------------------------------------------------------

\begin{minipage}[t]{0.33\textwidth} % The left column takes up 33% of the text width of the page

%------------------------------------------------
% Education
%------------------------------------------------

\section{Education} 

\subsection{UC Santa Cruz }

\descript{BS in Computer Science}
\location{Expected May 2019 | Santa Cruz, CA}
Conc. in Software Engineering \\
Jack Baskin School of Engineering \\
Dean's List (All Semesters) \\
\location{ Cum. GPA: 3.92 / 4.0}

\sectionspace % Some whitespace after the section

%------------------------------------------------

\subsection{La Canada High School}

\location{Grad. May 2015 | Pasadena, CA}

\sectionspace % Some whitespace after the section

%------------------------------------------------
% Links
%------------------------------------------------

\section{Links} 

Github:// \href{https://github.com/Leo-Neat}{\bf Leo-Neat} \\
LinkedIn:// \href{https://www.linkedin.com/in/leo-neat-82b34b130/}{\bf leo-neat} \\

\sectionspace % Some whitespace after the section

%------------------------------------------------
% Coursework
%------------------------------------------------

\section{Coursework}

\subsection{Graduate}

Computer Vision and Image Processing \\
Advanced Algorithm Analysis \\\
%------------------------------------------------
\subsection{Undergraduate}

Machine Learning \\
Advanced Programming \\
Comparative Programming Languages \\
Probability and Statistics \\
Computer Systems \\
Assembly Language \\
Discrete Math \\
Data Structures \\
Linear Algebra \\
Vector Calculus \\


\sectionspace % Some whitespace after the section

%------------------------------------------------
% Skills
%------------------------------------------------

\section{Skills}

\subsection{Programming}

\location{Over 5000 lines:}
Java \textbullet{} Python \textbullet{} Android \textbullet{} \LaTeX\ \\ 
\location{Over 1000 lines:}
C \textbullet{} C++ \\
\location{Familiar:}
TensorFlow's Object Detection API \textbullet{} Keras \textbullet{} OpenCV \textbullet{} SQLite \textbullet{} Shell  \textbullet{} Arduino

\sectionspace % Some whitespace after the section

%----------------------------------------------------------------------------------------

\end{minipage} % The end of the left column
\hfill
%
%----------------------------------------------------------------------------------------
%	RIGHT COLUMN
%----------------------------------------------------------------------------------------
%
\begin{minipage}[t]{0.66\textwidth} % The right column takes up 66% of the text width of the page

%------------------------------------------------
% Experience
%------------------------------------------------

\section{Experience}

\runsubsection{Aquifi}
\descript{| Software Engineering Intern}

\location{June 2018 – Sep 2018 | Palo Alto, CA}
\vspace{\topsep} % Hacky fix for awkward extra vertical space
\begin{tightitemize}
	\item 	Responsible for developing an Android application that allows users to quickly offload data from Aquifi devices to different servers for quality assurance, regression testing, training, etc...
	\item Wrote, tested, and modified system level camera code that ended up in production
	\item Developed an algorithm to detect corrupted frames in camera streams and recover the streams if necessary 
\end{tightitemize}

\sectionspace % Some whitespace after the section

%------------------------------------------------

\runsubsection{Jet Propulson Laboratory }
\descript{| Software Engineering Intern}

\location{Optical Software for WFIRST Mission | Summer 2017}
\begin{tightitemize}
	\item 	Designed and built a camera characterization system for EMCCD camera evaluation using an Android phone to emulate a star.
	\item 	The camera characterization system is currently being used by different teams at NASA for spaceflight detector evaluation 
	\item Saved NASA thousands of dollars in development cost and months of testing time 
\end{tightitemize}
\location{Computer Vision for Earth Facing Telescopes | Summer 2016}
\begin{tightitemize}
	\item  	Wrote code that implemented OpenCV's tracking methods in order to determine cloud heights from MISR (Multi-angle Imaging SpectroRadiometer) satellite images  
	\item 	Researched different forms of feature detection and optical flow in order to solve a stereo problem apparent in MISR’s images 
\end{tightitemize}
\location{Optical Software for WFIRST Mission | Summer 2015}
\begin{tightitemize}
	\item 	Developed code that reduced the execution time of the Wide-Field Infrared Survey Telescope (WFIRST) coronagraph testbed by a factor of 14
	\item 	Wrote an algorithm for detecting cosmic radiation in images from Electron Multiplying Charge Coupled Device (EMCCD) cameras 
\end{tightitemize}
\location{Software and Ground Truthing Intern | Summer 2014}
\begin{tightitemize}
	\item 	Wrote a Graphical User Interface which was utilized by JPL scientists for data manipulation and visualization of images 
	\item 	Developed master class for multi-core processing system for use in JPL computer lab 
\end{tightitemize}

\sectionspace % Some whitespace after the section
%------------------------------------------------
% Research
%------------------------------------------------

\section{Research}

\runsubsection{UC Santa Cruz Computer Vision Lab}
\descript{| Undergrad Research}

\location{December 2016 – Present | Santa Cruz, CA}
Worked with \textbf{\href{https://www.soe.ucsc.edu/people/manduchi}{Prof Roberto Manduchi}} and \textbf{\href{https://www.linkedin.com/in/siyang-qin-12b98467/}{Dr. Siyang Qin}} to create an Android application to be used by the visually impaired to identify text in their surroundings. The application streams image data from the phone's camera to a server which returns the location and characters of text found in the data sent to the server. 
Currently developing an Android application that is used to test and compare the inference speed of TensorFlow Lite models on mobile devices and studying the costs and benefits of on-board vs. server side inference for mobile Convolutional Neural Networks.

\sectionspace % Some whitespace after the section

%----------------------------------------------------------------------------------------

\end{minipage} % The end of the right column

%----------------------------------------------------------------------------------------
%	SECOND PAGE (EXAMPLE)
%----------------------------------------------------------------------------------------

\newpage % Start a new page

\begin{minipage}[t]{0.33\textwidth} % The left column takes up 33% of the text width of the page
\section{Projects}
\subsection{Crowd Size Detector}
	Developed a platform using Tensor-Flow's object detection API to monitor the number of people in a variety of public locations.
\sectionspace

\subsection{UCSC Hackathon}
	Wrote a Monte Carlo simulation to help predict the demand for each of the Computer Science Classes offered at UCSC. The program's goal was to help reduce the number of people unable to get into Computer Science classes.

\end{minipage} % The end of the left column
\hfill
\begin{minipage}[t]{0.66\textwidth} % The right column takes up 66% of the text width of the page


%------------------------------------------------
% Awards
%------------------------------------------------

\section{Publications} 

\begin{enumerate}
	\item
	L. Neat, R. Peng, S. Qin, R. Manduchi "Scene Text Access: A Comparison of Mobile OCR Modalities for Blind Users." 23rd International Conference on Intelligent User Interfaces. ACM, 2019.
	
	\item
	Michael Bottom, Leo S. Neat, Leon K. Harding, Patrick Morrissey, Seth R. Meeker, Richard T. Demers  "Smartphone scene generator for efficient characterization of visible imaging detectors", Proc. SPIE 10709, High Energy, Optical, and Infrared Detectors for Astronomy VIII, 107092R (6 July 2018); doi: 10.1117/12.2312335;
	
	\item
	Harding, et al.  ``Technology Advancement of the CCD201-20 EMCCD for the WFIRST-AFTA Coronagraph Instrument: sensor characterization and radiation damage.'' Journal of Astronomical Telescopes, Instruments,  and Systems 
	​ 2.1 (2016): 011007-011007. 
\end{enumerate}

\end{minipage} % The end of the right column

%----------------------------------------------------------------------------------------

\end{document}